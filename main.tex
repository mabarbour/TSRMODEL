\documentclass[11pt]{article}
\usepackage[sc]{mathpazo} %Like Palatino with extensive math support
\usepackage{fullpage}
\usepackage[authoryear,sectionbib,sort]{natbib}
\linespread{1.7}
\usepackage[utf8]{inputenc}
\usepackage{lineno}

%%%%%%%%%%%%%%%%%%%%%
% LaTeX packages
%%%%%%%%%%%%%%%%%%%%%
% Please be sparing in your use of additional LaTeX packages, and
% upload any required style files to Editorial Manager with the file
% type "LaTeX ancillary files (.sty, .bst)."
\usepackage{amsmath}
\usepackage{graphicx} %remove when remove figures
\graphicspath{ {IMAGES/} }

%%%%%%%%%%%%%%%%%%%%%
% Line numbering
%%%%%%%%%%%%%%%%%%%%%
\usepackage{lineno}
% Please use line numbering with your initial submission and
% subsequent revisions. After acceptance, please comment out 
% the commands \usepackage{lineno}, \linenumbers{} 
% and \modulolinenumbers[3] below.

\title{The temperature-size rule is predicted to stabilize the response of consumer-resource dynamics under warming\\
or\\
Temperature-dependent body size alters the effects of temperature on consumer resource dynamics}

%%%%%%%%%%%%%%%%%%%%%
% Authorship
%%%%%%%%%%%%%%%%%%%%%
% Please remove authorship information while your paper is under review,
% unless you wish to waive your anonymity under double-blind review. 
% Remember to uncomment the information after acceptance.

%\author{Matthew Miles Osmond$^{1,\ast}$ \\ 
%et al$^{2}$}

\date{}

\begin{document}

\maketitle

%\noindent{}1. Biodiversity Research Centre and Department of Zoology, University of British Columbia, Canada;
%
%\noindent{}2. others;
%
%\noindent{}$\ast$ Corresponding author; e-mail: mmosmond@zoology.ubc.ca.

\bigskip

\textit{Manuscript elements}: Figure~1, figure~2, figure~3, supplementary \texttt{Mathematica} file.
\bigskip

\textit{Keywords}: Metabolic theory, predator-prey, plant-herbivore, body size, allometry, functional response, mathematical model.

\bigskip

\textit{Manuscript type}: Note. 
% Or e-article, note, e-note, natural history miscellany,
% e-natural history miscellany, comment, reply, symposium, or
% countdown to 150.

\bigskip

\noindent{\footnotesize Prepared using the suggested \LaTeX{} 
template for \textit{Am.\ Nat.}}

\linenumbers{}
\modulolinenumbers[3]

\newpage{}

%%%%%%%%%%%%%%%%%%%%%%%%%%%%%%%%%%%%%%%%%%%%%%%%
\section*{Abstract}
%%%%%%%%%%%%%%%%%%%%%%%%%%%%%%%%%%%%%%%%%%%%%%%%
Both body size and temperature directly influence the dynamical relationship between consumers and their resources. There is also a widespread negative relationship between temperature and size; this is known as the temperature-size rule (TSR). The growing theory on temperature-dependent consumer resource interactions has yet to integrate the TSR into a general framework for how temperature affects consumer resource dynamics. 
We expanded an existing temperature-dependent consumer-resource model to include the indirect effects of warming, through changes in body size, and parameterized the model with values drawn from data syntheses. 
We analyzed this model to answer the following questions: 
1) How does including the TSR affect predictions for how temperature affects consumer-resource stability and biomass ratios? 
2) Under what circumstances are the effects of the TSR most substantial? 
%3) Is the TSR predicted to induce a change in the functional response? 
We found that including the TSR led to two qualitatively different predictions: under warming i) consumer-resource biomass is no longer expected to decline and ii) the dynamics are expected to become more stable, as opposed to the decline in stability predicted without the TSR. 
These qualitatively different predictions were strengthened by asymmetric temperature-size responses between consumers and resources and type-II functional responses.
Our analyses suggest that the effect of temperature on body size likely plays an important role in the response of consumer-resource systems to changing temperatures. 

\newpage{}

%%%%%%%%%%%%%%%%%%%%%%%%%%%%%%%%%%%%%%%%%%%%%%%%
\section*{Introduction}
%%%%%%%%%%%%%%%%%%%%%%%%%%%%%%%%%%%%%%%%%%%%%%%%

% The journal does not have numbered sections in the main portion of
% articles. Please refrain from using section references such as
% section~\ref{section:CountingOwlEggs}, and refer to sections by name
% (e.g. section ``Counting Owl Eggs'').

% Please note that we prefer (\citealt{Xiao2015}) to \citealt{Xiao2015},
% since \citealt{} inserts a comma after "et al."

%%%%%%%%%%%%%%%%%%%%%%%%%%%%%
%my little weekend attempt

%Temperature and body size determine many biological rates (\cite{West1997,Gillooly2001}).
%How these factors individually influence consumer-resource dynamics has already been demonstrated (\cite{Gilbert2014,DeLong2015}).
%How they act in symphony has not yet been investigated.
%
%Temperature also affects body size through the temperature-size rule (\cite{Atkinson1994}).
%Thus, temperature directly and indirectly affects ecosystem dynamics.
%These two pathways have the potential to reinforce or counteract one another, with population level consequences.
%
%Here we model a simple consumer-resource interaction, with population dynamic parameters that depend on temperature and body size, and body sizes that depend on temperature.
%We ask how including dependencies on both temperature and body size, as well as incorporating the temperature-size rule, affects the response of the consumer-resource dynamics to increasing temperature.

%%%%%%%%%%%%%%%%%%%%%%%%%%%%%

Populations of consumers and their resources are joined across time by the fact that energy to the consumer comes from the resource, and the consumer determines mortality rates of the resource. 
In these systems, temperature-dependent consumption, growth and mortality rates can change dynamics and their outcomes. 
Through the temperature-dependence of metabolism \citep{West1997,Gillooly2001} and hence temperature-dependent demographic vital rates, small changes in temperature that are not necessarily physiologically stressful for organisms can translate to changes in the stability and coexistence of consumers and their resources, producing predictable effects of warming on simple food webs \citep{Gilbert2014,Vasseur2005,OConnor2011,Rall2010}. 

Demographic rates and consumer-resource interactions also depend on body size \citep{Yodzis1992,DeLong2015}. 
Not only do rates of growth, mortality and consumption scale predictably with individual body sizes, but the consumption rates of consumer-resource systems can also depend on the ratio of body masses between consumers and prey \citep{Kalinkat2013}. 
Changes in body size or the ratio of body sizes between interacting species be can therefore change demographic and interaction rates. Given the importance of body size to the dynamics and outcomes of consumer-resource interactions, frameworks for understanding how temperature affects consumer-resource dynamics have not considered the importance of changes in body size with temperature.

The frequently-observed negative relationship between temperature and body size, the temperature-size rule (TSR), has been called the third universal response to warming \citep{Gardner2011}. 
% * <mwpennell@gmail.com> 2016-06-24T21:49:24.120Z:
%
% pedantic point: does the TSR describe a response to changes in temperature or simply a negative relationship between the two
%
% ^ <mwpennell@gmail.com> 2016-06-24T21:51:49.434Z:
%
% and also: what are the other two? seems odd to mention this as the third without that context
%
% ^.
For comparisons within populations, among species and across biogeographic gradients, body size of ectotherms tends to decline with increasing temperature \citep{Atkinson1994,Daufresne2009,Forster2012,DeLong2012}. 
Although the mechanism for declining body size with warming varies among examples, including physiological plasticity, selection for smaller individuals, and turnover in species composition, the pattern is similar across levels of organization \citep{Forster2012}. 
\citet{Forster2012} reported a mean slope of -3.65$\%/ ^\circ C$ for aquatic organisms, ranging from -1.80$\% / ^\circ C$ for unicells, and becoming stronger (more negative) in increasingly large aquatic multicelled organisms.

Because body size is so central to consumer-resource dynamics, such a systematic pattern of changing body size with temperature could alter predictions for how temperature affects stability, persistence and coexistence in consumer-resource systems. 
We therefore integrated the TSR into a general framework for temperature-dependent consumer-resource interactions to answer the following questions: 
1) How does the TSR affect stability and consumer:resource biomass ratios over a temperature gradient?, 
2) Does the effect of the TSR depend on whether consumer and resource body sizes respond similarly to temperature?, 
3) Does the effect of the TSR depend on the form of the functional response?, 
And finally, 4) does the TSR itself induce a change in the functional response?

%%%%%%%%%%%%%%%%%%%%%%%%%%%%%%%%%%%%%%%%%%%%%%%%%%%
\section*{Methods and results}
%%%%%%%%%%%%%%%%%%%%%%%%%%%%%%%%%%%%%%%%%%%%%%%%%%%

%%%%%%%%%%%%%%%%%%%%%%%%%%%
\subsection*{The underlying consumer-resource dynamics}
%As an aside, I've always found it interesting that the Rosenzweig-MacArthur equations are attributed to this paper, although I've never been able to clearly identify these equations in the paper.
We begin, like \cite{Gilbert2014}, with the Rosenzweig-MacArthur equations \citep{Rosenzweig1963}
\begin{equation}\label{eq:RM}
\begin{aligned}
\frac{\mathrm{d}R}{\mathrm{d}t} =& r R \left(1 - \frac{R}{K} \right) - f(R) R C\\
\frac{\mathrm{d}C}{\mathrm{d}t} =& e f(R) R C - m C,
\end{aligned}
\end{equation}
which describe the rates of change in total resource $R\in[0,K]$ and consumer $C\geq0$ biomass with time $t$.

In the absence of consumers, $C=0$, the resource grows logistically, with intrinsic growth rate $r\geq0$ and carrying capacity $K>0$.
The intrinsic growth rate describes the rate at which resource biomass increases (per unit biomass) in the absence of consumers when the resource is rare, $R\approx0$.
The carrying capacity is the equilibrium biomass of the resource without consumers.

Resource biomass is consumed by consumers at a rate $f(R) R C$, where $f(R)\geq0$ is called the functional response.
Of the biomass consumed, the unitless conversion efficiency parameter $e\in[0,1]$ determines the proportion of resource biomass that is directly converted into consumer biomass.
Consumers biomass dies at a constant per unit biomass mortality rate $m\geq0$.

An equilibrium is reached when the two rates of change in Equation \eqref{eq:RM} are zero, and solving the system at this point gives equilibrium resource $\hat{R}$ and consumer $\hat{C}$ biomass.
There are three equilibria for this system: total extinction $(R,C) = (0,0)$, consumer extinction $(R,C)=(K,0)$, and coexistence $(R,C)=(\hat{R},\hat{C})$, with $\hat{R}>0$ and $\hat{C}>0$.
We are primarily concerned with the latter equilibrium, as that is presumably the equilibrium current consumer-resource systems are near.
At this coexistence equilibrium one can calculate the ratio of consumer to resource biomass, $\hat{C}:\hat{R}$, and also perform a linear stability analysis to derive the leading (largest in absolute value) eigenvalue $\lambda$, which determines if (and how readily) the system, when perturbed a small amount from this equilibrium, will return to it (see the supplementary \texttt{Mathematica} file for details).
Our measure of stability will be the negative of the real part of the leading eigenvalue.
%I think I'm getting confused by the wording here in lines 170 and 171. For a continuous time model such as this one, doesn't a negative eigenvalue correspond to a stable system?
The system is stable if and only if this value is positive, and the system will return to equilibrium faster when this value is larger (i.e., larger positive values imply ``more stable" systems).
Together these two measures tell us how biomass is partitioned and how stable this partitioning is.

As explained in \cite{Gilbert2014}, two aggregate parameters well describe the dynamics of this system.
The first is $m /(e f(\hat{R}))$, which describes (the inverse of) consumer growth at equilibrium, is the slope of the consumer zero-net growth isocline, and is the abundance of the resource at the coexistence equilibrium.
The second aggregate is $K$, the equilibrium resource biomass in the absence of consumers.
Dividing the second aggregate by the first gives a measure that defines the biomass potential of the resource that is converted into consumer biomass \[B_{CR} = \frac{e f(\hat{R}) K}{m}.\]
%Note that we have modified the presentation of \cite{Gilbert2014} by subsuming $R$ into $f(R)$ such that $f(R)$ is  a rate and thus more easily interpreted.
%This does not alter the results.

In what follows we will examine how our three measures, $B_{CR}$, $\hat{C}:\hat{R}$, and stability, change with temperature.
We start by assuming a type-I functional response, $f(R) = a$, where $a$ is called the attack rate, which describes the rate of resource consumption per resource biomass.
We later explore the effect of a type-II functional response and the potential for the functional response to change with changes in temperature.

%%%%%%%%%%%%%%%%%%%%%%%%%%%
\subsection*{Adding temperature dependence}

\cite{Gilbert2014} discuss what is known about the temperature dependencies of the population dynamic parameters $r$, $K$, $a$, $m$, and $e$, and give equations and parameter estimates (from previously published sources) in their Table 1.
Briefly, resource growth rate $r$ is expected to scale with metabolism as a Boltzmann-Arrhenius factor, \[r(T) = r_0 \exp(-E_B/(kT))\] where $E_B$ is the activation energy of metabolism $B$ (in units of eV), $k$ is Boltzmann's constant ($\approx 8.62 \times 10^{-5}$ eV/Kelvin), and $T$ is the temperature (in Kelvins).
Resource carrying capacity $K$ is determined by the ratio of the supply rate of nutrients into the system, S, and the rate of uptake of nutrients by the resource, $r$.
With supply rate also scaling as a Boltzmann-Arrhenius factor with activation energy $E_S$, the prediction for carrying capacity becomes  \[K(T) = K_0 \exp(-(E_S - E_B)/(kT)).\] 
Attack rate $a$ depends on the temperature dependence of the body velocities $\nu$ in both species, both of which scale as Boltzmann-Arrhenius factors with activation energies $E_{\nu,i}$, for $i=\{R,C\}$.
Attack rate is then \[a(T) = a_0 \sqrt{\sum_i \left[\nu_{0,i} \exp(-E_{\nu,i}/(kT)) \right]^2}\] where $\nu_{0,i}$ are rate-constants.
Consumer mortality is also expected to scale as a Boltzmann-Arrhenius factor, \[m(T) = m_0 \exp(-E_m/(kT)).\]
Conversion efficiency is assumed to be independent of temperature such that $e(T) = e_0$.

The black curves in Figure \ref{AllTempMassDep} show $B_{CR}$, equilibrium consumer to resource biomass ratio $\hat{C}:\hat{R}$, and stability of the coexistence equilibrium as functions of temperature $T$ (plotted in Celsius).
In Figure 3 of \citet{Gilbert2014}, only $K$ depends on temperature; the rest are held constant. If we consider all of the additional dependencies discussed by \citet{Gilbert2014}
In these plots $r$, $K$, $a$, and $m$ all depend on temperature, unlike Figure 3 in \cite{Gilbert2014} where only $K$ depends on temperature and the rest of the parameters are held constant.
Comparing with Figure 3 in \cite{Gilbert2014}, we see that adding temperature dependence in $r$, $a$, and $m$ causes equilibrium consumer:resource biomass to decline with temperature (instead of increasing) and stability to decrease at a slower rate  with increasing temperature.
These changes are largely driven by the temperature dependence of consumer mortality (see supplementary \texttt{Mathematica} file): increasing temperature increases consumer mortality, lowering equilibrium consumer biomass and increasing stability at high temperatures (relative to the case where mortality $m$ does not depend on temperature).
% * <barbour@zoology.ubc.ca> 2016-06-24T13:58:58.890Z:
%
% MO: why do you attribute the decline in C:R biomass and relatively greater stability to higher consumer mortality (primarily)? Is the logic something like: increasing m, by itself, would create this pattern; increasing r, by itself, would increase C:R, but also increase stability (at least in type 1); and increasing a, by itself, would increase C:R, but decrease stability? I'm mainly asking to also check my understanding of the influence of different parameters on C-R models.
%
% ^.
%no, this isn't a logic thing at all but simply me including or excluding the temperature dependence of parameters in the mathematica file until i find which one is driving the pattern. so here for example, when i make m temperature-independent the prediction looks much like gilberts, and dropping temp dependence of the other parameters does not affect things as much. (MMO)

%%%%%%%%%%%%%%%%%%%%%%%%%%%
\subsection*{Adding mass dependence and the temperature size-rule}

We next allow the population dynamic parameters to depend on the body size of the interacting species.
Following \cite{DeLong2015}, each parameter can be written as a power law function of the body mass of resource $M_R$ or consumer $M_C$.
Here we combine \cite{DeLong2015} and \cite{Gilbert2014} by letting the parameters depend on both temperature and mass: $r(T, M_R) = r(T) M_R^\rho$, $K(T, M_R) = K(T) M_R^\kappa$, $a(T, M_C) = a(T) M_C^\alpha$, $e(T, M_C) = e(T) M_C^\epsilon$, and $m(T, M_C) = m(T) M_C^\mu$.

If mass does not change with temperature then adding these mass dependencies does not change the response of the consumer-resource dynamics to temperature.
However, mass is expected to change with temperature, according to the temperature-size rule (\cite{Atkinson1994}).
%For organisms with a dry mass of less than $10^{-3}$ mg body mass declines linearly with temperature.
We incorporate a simple form of the temperature-size rule here for illustrative purposes.
In particular, we assume body mass declines linearly with temperature, \[M_i(T) = M_i(T_{ref}) (1 - \beta_i (T - T_{ref})) \], where $\beta_i$ is the fraction that mass is reduced as temperature is increased by one degree and $T_{ref}$ is a reference temperature, which we set to 15$^\circ C$ throughout.
This linear decline best approximates the response of organisms with a dry mass of less than $10^{-3}$ mg, whereas larger organisms experience a faster than linear decline \citep{Forster2012}.

The red curves in Figure \ref{AllTempMassDep} depict our main results: adding mass dependencies and the temperature-size rule modifies our prediction of how consumer-resource dynamics respond to changes in temperature.
While there is little change in $B_{CR}$, the equilibrium consumer to resource biomass ratio is no longer expected to decline with increasing temperature and stability is now expected to increase.
These changes are brought on by the indirect effect of temperature, through body mass, on the population dynamic parameters, which are acting in opposition to its direct effects.
In particular, the lack of decline in the consumer to resource biomass ratio with the temperature-size rule, relative to the case without it, is primarily driven by changes in consumer conversion efficiency and the intrinsic growth rate of the resource (see supplementary \texttt{Mathematica} file).
Both of these rates increase with declining body mass, supporting a relatively larger consumer biomass. 
The increase in stability at high temperatures with the temperature-size rule is caused by the increase in the resource's intrinsic growth rate along with a decrease in attack rate with decreasing consumer body size. 
% * <barbour@zoology.ubc.ca> 2016-06-24T13:58:32.133Z:
%
% It looks like mortality rate would also increase with declining body mass, won't this also contribute to increased stability (although I understand that it would decrease C:R)
%
% ^.
%logic is like that explained above: i just drop the mass dependencies of each parameter in turn and see which ones are having the largest effect (see mathematica file) (MMO)
%
%The indirect effects of temperature are acting in opposition to its direct effects.
So we see that, in the case of stability, the indirect effects of temperature are strong enough to override its direct effects, producing a qualitatively different prediction of how consumer-resource systems will respond to temperature.
%The increase in stability with the temperature-size rule stimulates the tantalizing possibility that its existence is adaptive, although obviously much more careful thought is needed here.

In the supplementary \texttt{Mathematica} file we explore how the strength of the temperature-size response $\beta_C = \beta_R = \beta$ affects our predictions (see also Figure \ref{StrengthAsymm}, green). 
We find that predictions for biomass ratio and stability at higher temperatures differ qualitatively from those of \cite{Gilbert2014} for $\beta\geq0.02$.
The biomass ratio begins to increase with temperature around $\beta\sim0.03$.
Larger temperature-size responses cause both the biomass ratio and stability to increase faster with temperature. %(and decrease faster with declining temperature).

%%%%%%%%%%%%%%%%%%%%%%%%%%%
\subsection*{Exploring asymmetric temperature-size responses}

In Figure \ref{AllTempMassDep} we assumed both resource and consumer body mass declined with temperature at the same rate, $\beta_C = \beta_R = 0.02$, i.e., both decline $2\%$ per degree increase.
However, larger organisms often experience larger declines in body size with temperature (\cite{Forster2012}).
%In Figure \ref{AllTempMassDepAsymm} we let consumer body size decline twice as fast as the resource, $2 \beta_R = \beta_C = 0.04$.
In Figure \ref{StrengthAsymm} (blue) we let consumer body size decline twice as fast as the resource, $2 \beta_R = \beta_C = 0.04$.
The main effect of the asymmetric temperature-size response is that i) the $B_{CR}$ with $E_S > E_B$ now asymptotes at higher temperatures (compare with the dark solid curves in Figure \ref{AllTempMassDep}) and ii) stability now increases even faster with increasing temperature.
These effects are driven by the now larger decline in attack rate.
Thus, expected asymmetries in the temperature-size response cause our predictions to deviate even further from those of without the temperature-size rule \citep{Gilbert2014}.

%%%%%%%%%%%%%%%%%%%%%%%%%%%
\subsection*{Type-II functional response}

The consumption of resources in some systems may be better described by a type-II functional response, \[f(R) = b / (1 + b h R)\] where $b$ is sometimes called the capture rate (the per resource biomass per consumer biomass rate of resource biomass consumption) and $h$ is the handling time.
This collapses to a type-I functional response at low resource biomass, $f(R) \approx b$ for $R << 1/(b h)$.
At high resource biomass a type-II functional response implies that the rate of resource consumption per consumer biomass asymptotes at $\lim_{R\rightarrow\infty}f(R) R = 1/h$, describing satiation of the consumer.

Both capture rate and handling time are known to depend on temperature and body mass.
In particular, \cite{Rall2012} argue that capture rate scales like 
\[b(T, M_R, M_C) = b_0 M_R^{b_R} M_C^{b_C} \exp(-E_b/(k T))\]
and handling time like 
\[h(T, M_R, M_C) = h_0 M_R^{h_R} M_C^{h_C} \exp(-E_h/(k T))\].
With these scaling the type-II functional response increases faster with temperature than the type-I functional response (see supplementary \texttt{Mathematica} file) %shows how the functional response changes with temperature.
Much of the difference in the response of the functional responses to temperature is due to the differing temperature- and mass- dependencies of attack rate $a$ and capture rate $b$ (see supplementary \texttt{Mathematica} file), and not due to the form of the functional responses (i.e., setting $h=0$ has little effect on the response of the type-II functional response to temperature).
When capture rate has the same temperature- and mass- dependencies as attack rate (\cite{Gilbert2014}), there is little difference between the response of the type-I and type-II functional responses to temperature (see supplementary \texttt{Mathematica} file).

With the parameter values given in \cite{Rall2012}, we can plot $B_{CR}$, $\hat{C}:\hat{R}$, and stability as functions of temperature when there is a type-II functional response (Figure \ref{TypeII}).
The main conclusions are: in comparison to a type-I functional response, a type-II functional response i) makes $B_{CR}$ increase faster with temperature, ii) makes equilibrium biomass ratios increase with temperature, and iii) makes stability increase more quickly with temperature, despite the fact that a type-II functional response decreases stability at our reference temperature (15$^\circ$C, i.e., without a direct or indirect temperature response).
Thus, a type-II functional response, like asymmetric temperature-size responses, causes our predictions to vary further from those without the temperature-size rule \citep{Gilbert2014}.
% * <barbour@zoology.ubc.ca> 2016-06-24T13:57:51.700Z:
%
% MO: just to clarify, was your exploration of the type 2 functional response always restricted to cases where the equilibrium was stable? I ask, because according to McCann (2012, book), increasing intrinsic growth rate has qualitatively different effects on stability depending on whether the consumer isocline is to the left or right of the resource's hump-shaped isocline. In other words, if the system is exhibiting a stable limit cycle, then increasing intrinsic growth rate will actually further destabilize the system.
%
% ^.
% yes, always where stable (i chose handling time such that this was true)

As an aside, in this analysis we set $b_0$ to give $f(R) = 0.1$ at 15$^\circ C$, to remain consistent with \cite{Gilbert2014}.
This leaves $h_0$ as a free parameter. 
However, this free parameter only influences the results in the case of stability.
When $h_0$ is small enough to allow a stable coexistence equilibrium (roughly $h_0 <10^{-12}$) we find that stability increases exponentially with temperature (blue curve in Figure \ref{TypeII}).
Thus, even though a type-II functional response decreases stability at the reference temperature (because of the lags induced by handling time), stability is increased at higher temperatures. 
The increased stability at higher temperatures with a type-II functional response is caused by the temperature dependence of the capture rate (\cite{Rall2012}), which differs from the temperature dependence of the type-I attack rate \citep{Gilbert2014}.
Giving capture rate the same temperature dependence as attack rate \citep{Gilbert2014}, with a small enough handling time at the reference temperature ($h_0 <\sim10^{-13}$) stability still increases exponentially with temperature, from the reference temperature, but the square root in the expression now allows the temperature dependence of other parameters to slow and revert this increase at higher temperatures.
Larger handling times prevent the exponential increase in stability with temperature, and the increase in handling time with temperature can even cause stability to decrease with temperature (see supplementary \texttt{Mathematica} file for details). 

\cite{Rall2012} complied a large database on capture rates and handling times and compared the data to their theoretical predictions.
They found that capture rate and handling time responded less strongly to temperature than expected (see their Figure 2a,d).
Interestingly, we find that the temperature-size rule reduces the sensitivity of both capture rate and handling time to temperature (see supplementary \texttt{Mathematica} file) and hence may help explain the discrepancies observed. 

%%%%%%%%%%%%%%%%%%%%%%%%%%%
\subsection*{A functional response that depends on the body size ratio}

Functional responses tend to be roughly type-II when consumers and resources have similar body sizes, but become more sigmoidal and hence more type-III when consumers are much bigger than their resource, as resources are then better able to hide when rare \citep{Kalinkat2013}.
Without the temperature-size rule, the body mass ratio remains constant with temperature, and therefore the form of the functional response is not expect to change.
However, with the temperature-size rule, body sizes change.
When the temperature-size responses are asymmetric, the ratio of body sizes will change and influence the form of the functional response.
Because consumers are often larger than their resource, and because larger organisms are expected to have greater reductions in body size with temperature \citep{Forster2012}, the ratio of consumer to resource body size will often decrease with temperature.
As stated above, lower consumer to resource body size ratios produce functional responses more like type-II, which are less stable than type-III functional responses.
Hence the temperature-size rule can be said to destabilize the consumer-resource dynamics at high temperatures by promoting type-II functional responses (it can also be said that the temperature-size rule stabilizes the dynamics at lower temperatures by promoting type-III functional responses).
However, the amount by which the shape of the functional response is adjusted by the temperature-size rule does not appear to be large and therefore the stabilizing effects discussed in previous sections will likely prevail (see supplementary \texttt{Mathematica} file for details). 

%%%%%%%%%%%%%%%%%%%%%%%%%%%%%%%%%%%%%%%%%%%%%%%%%%
\section*{Discussion}
%%%%%%%%%%%%%%%%%%%%%%%%%%%%%%%%%%%%%%%%%%%%%%%%%%%

Our results suggest changes in body size with warming have important dynamical consequences for food webs in changing thermal environments. 
In our temperature- and mass-dependent consumer-resource model, declining body size with warming changes the predicted outcome of consumer-resource dynamics in response to temperature. 
When body size declines with warming at a rate consistent with empirical observations (\cite{Forster2012}), consumer-resource biomass ratios remain stable under warming. 
Further, a TSR response facilitates a slight increase in system stability with warming, relative to a decrease in stability without shifts in body size. 
At the same time, the bioenergetic interaction strength metric (BCR) is relatively unaffected by the TSR. 

%When consumer and resource body sizes decline (proportionally) at the same rate with warming, there is no shift in the consumer to resource body size ratio over the temperature gradient. 
%In this case, deviations in how consumer and resource biomass respond to temperature, compared with a system with no TSR, reflects the body mass-dependence of demographic parameters. 
Deviations in the response of consumer-resource dynamics to temperature, compared to a system with no TSR, reflects the body mass-dependence of demographic parameters. 
In this way, temperature acts on the system first by directly altering demographic rates, and then by indirectly altering demographic rates through allometric scaling relationships. 
For instance, the indirect response of the resource to warming - increased intrinsic growth rates at smaller body sizes - allows the resource to support a relatively greater consumer biomass at higher temperatures if body sizes decline. 
When we allowed larger (consumer) organisms to experience a greater decline in body size with warming, and thus a temperature-dependent consumer-resource body size ratio, effects of the TSR were further amplified relative to a model with symmetric declines in body size. %this last sentence could start a new paragraph? (MMO)

The body size decline rates used here were drawn from a comprehensive synthesis of experimentally observed temperature-body size responses in organisms ranging from fish to microbes (\cite{Forster2012}). 
Although there are a number of theoretical explanations for the TSR (\cite{Berrigan1994,Perrin1995,VanderHave1996,Angilletta2003,DeLong2012}), much of its support is empirical, and rates of decline are not yet predictable a priori.
Drawing on this empirical evidence, we considered effects of instantaneous body size shifts within species. 

%Limitations
This model does not consider shifts in species composition that could further compound the dynamical effects of shifts in body size. 
Our model also assumes that body size responds to temperature independently of any effects of the changing consumer-resource interaction on body size. 
In other words, consumer-resource interactions can select for optimal body sizes (\cite{Abrams1996}), and this selection may vary over a thermal gradient. 
%But here, we only modelled a decline in size due to some mechanisms that is independent of CR interaction.
% * <joey.bernhardt@biodiversity.ubc.ca> 2016-06-24T04:19:49.804Z:
%
% > In other words, consumer-resource interactions can select for optimal body sizes (REF), and this selection may vary over a thermal gradient. But here, we only modeled a decline in size due to some mechanisms that is independent of CR interaction.
%
% yes, but can we say this? I think we can only say that the cause of the body size decline that we are modeling could be one of many (i.e. may or may not be due to a CR interaction since the lit reviews didn't control for this, right?)
%
% ^.
% * <barbour@zoology.ubc.ca> 2016-06-24T13:59:21.350Z:
%
% I'm thinking it could be useful to have a paragraph that discusses how our results connect more generally to consumer-resource theory. All of the parameters in the C-R models, when varied independently, have predictable effects on C:R biomass ratios and stability (at least according to Kevin McCann's book).  For example, both the direct and indirect effects of increasing temperature result in increasing consumer mortality, which will decrease C:R ratios and increase stability. However, temperature simultaneously affects all of these parameters in the model (a point that perhaps wasn't clear in Gilbert et al. 2014, due to Fig. 3). In this scenario, the relative changes in these parameter values will govern the dynamics of the system. Based on the published syntheses which our study is based on, it seems like intrinsic growth rate is the most sensitive of the C-R parameters to the direct and indirect effects of temperature, and so is driving our findings that C:R ratios can increase and stay high, while stability is actually decreasing. Therefore, I think a job for empiricists is to measure the relative changes in the parameters for their system, because this will aid predictions about how increasing temperature will affect C:R biomass and stability. On a related note, the effects of increasing r (which is an important driver of the patterns we are seeing) depends on the geometry of the C:R isoclines. At the very least, I think we should make a point to say that increases in r, may actually destabilize a system that is exhibiting oscillatory dynamics, whereas it tends to increase stability when the system has a stable equilibrium.
%
% ^.
%[someone want to take it from here?] 
We have extended a general model for how temperature affects consumer-resource interactions by including a major empirical pattern, the TSR. 
Further extensions could include integrating multiple trophic levels to determine how temperature and the TSR affect trophic cascade strength. 
Our model suggests it would weaken it. %[do we really want to go here?]. 

%Possible experiment to test the effect of the TSR:
Our main predictions are: equilibrium consumer to resource biomass ratio is expected to increase with temperature (due to increases in consumer conversion efficiency and the growth rate of the resource associated with declining body sizes). Could test this by comparing two systems: one where body size is allowed to decrease with temperature, and another where the smallest individuals are removed (or the opposite). 

%talk about which areas are ripe for experimental exploration -- temperature dependencies of handling time/capture rate vs. magnitude and potential asymmetry of TSR responses


%%%%%%%%%%%%%%%%%%%%%
% Acknowledgments
%%%%%%%%%%%%%%%%%%%%%
% You are encouraged to remove the Acknowledgments section while
% your paper is under review (unless you wish to waive your anonymity
% under double-blind review) if the Acknowledgments reveal your
% identity. If you remove this section, you will need to add it back
% in to your final files after acceptance.

%\section*{Acknowledgments}
%Stilianos!

%%%%%%%%%%%%%%%%%%%%%%%%%%%%%%%%%%%%%%%%%%%%%%%%%%%
\bibliographystyle{amnatnat}
\bibliography{library}


\newpage{}

%%%%%%%%%%%%%%%%%%%%%
% Figure legends
%%%%%%%%%%%%%%%%%%%%%
% Please include all figure legends in a separate section at the end of 
% the document. If you use \label{} and \ref{} to refer to your figures,
% these can still work even if you comment out the %includegraphics{}
% line. If you refer to figures as "fig. 1" (etc.) manually, the
% legends can also appear simply as paragraphs.
% For submission, please upload the relevant figure files separately to
% Editorial Manager; Editorial Manager should insert them at the end of
% the PDF automatically.
% Figure legends should be concise, though they can be longer than the
% titles of tables.

\section*{Figure legends}

%%%%%%%%%%%%%%%%%%%%%%%%%%%%%%%%%%%%%%%%%%%%%%%%
%\begin{figure}[!ht]
%\centering
%\includegraphics[width=0.5\linewidth]{BCRAllTempDep}
%\includegraphics[width=0.5\linewidth]{CtoRAllTempDep}
%\includegraphics[width=0.5\linewidth]{StabilityAllTempDep}
%\caption{
%$B_{CR}$, equilibrium consumer to resource biomass ratio $\hat{C}:\hat{R}$, and stability of the coexistence equilibrium as functions of temperature $T$ (plotted in Celsius).
%Rate-constants (e.g., $r_0$) were chosen to make $r = 2$, $K = 100$, $a = 0.1$, $m = 0.6$, and $e = 0.15$ at 15$^\circ C$ (as in \cite{Gilbert2014}).
%Other parameters: $E_B = 0.32$ (solid black), $E_B = 0.9$ (dashed and gray), $E_S = 0.9$ (solid black and gray), $E_S = 0.32$ (dashed), $E_m = 0.65$, $E_{\nu,i} = 0.46$, $\nu_{0,i} = 1$.  
%}
%\label{AllTempDep}
%\end{figure}

%%%%%%%%%%%%%%%%%%%%%%%%%%%%%%%%%%%%%%%%%%%%%%%%
\begin{figure}[!ht]
\centering
\includegraphics[width=0.5\linewidth]{BCRAllTempMassDep}\\\vspace{-0.75cm}
\includegraphics[width=0.5\linewidth]{CtoRAllTempMassDep}\\\vspace{-0.75cm}
\includegraphics[width=0.5\linewidth]{StabilityAllTempMassDep}
\caption{
$B_{CR}$, equilibrium consumer to resource biomass ratio $\hat{C}:\hat{R}$, and stability of the coexistence equilibrium as functions of temperature $T$ (plotted in Celsius) with (red) and without (black) mass dependencies and the temperature-size rule.
Rate-constants were chosen to make $r = 2$, $K = 100$, $a = 0.1$, $m = 0.6$, and $e = 0.15$ at 15$^\circ C$ \citep[as in Figure 3 of][]{Gilbert2014}.
Other parameters as in \cite{Gilbert2014} and \cite{DeLong2015}: $E_B = 0.32$, $E_S = 0.9$, $E_m = 0.65$, $E_{\nu,i} = 0.46$, $\nu_{0,i} = 1$, $\kappa = -0.81$, $\alpha = 1$, $\epsilon = -0.5$, $\mu = -0.29$, $\rho = -0.81$, $\beta_i = 0$ (black), $\beta_i = 0.02$ (red).  
Note that we allow all population dynamic parameters to depend on temperature and mass, unlike Figure 3 in \cite{Gilbert2014}, where only $K$ varies.
}
\label{AllTempMassDep}
\end{figure}

%%%%%%%%%%%%%%%%%%%%%%%%%%%%%%%%%%%%%%%%%%%%%%%%
\begin{figure}[!ht]
\centering
\includegraphics[width=0.5\linewidth]{Figure2A}\\\vspace{-0.75cm}
\includegraphics[width=0.5\linewidth]{Figure2B}\\\vspace{-0.75cm}
\includegraphics[width=0.5\linewidth]{Figure2C}
\caption{
$B_{CR}$, equilibrium consumer to resource biomass ratio $\hat{C}:\hat{R}$, and stability of the coexistence equilibrium as functions of temperature $T$ (plotted in Celsius) without the temperature-size rule (black), with a relatively weak, symmetric temperature size rule (red), with an asymmetric temperature-size rule (green), and with a relatively strong, symmetric temperature-size rule (blue).
Rate-constants were chosen to make $r = 2$, $K = 100$, $a = 0.1$, $m = 0.6$, and $e = 0.15$ at 15$^\circ C$ \citep[as in Figure 3 of][]{Gilbert2014}.
Other parameters as in \cite{Gilbert2014} and \cite{DeLong2015}: $E_B = 0.32$, $E_S = 0.9$, $E_m = 0.65$, $E_{\nu,i} = 0.46$, $\nu_{0,i} = 1$, $\kappa = -0.81$, $\alpha = 1$, $\epsilon = -0.5$, $\mu = -0.29$, $\rho = -0.81$.
}
\label{StrengthAsymm}
\end{figure}

%%%%%%%%%%%%%%%%%%%%%%%%%%%%%%%%%%%%%%%%%%%%%%%%
\begin{figure}[!ht]
\centering
\includegraphics[width=0.5\linewidth]{BCRTypeII}\\\vspace{-0.75cm}
\includegraphics[width=0.5\linewidth]{CtoRTypeII}\\\vspace{-0.75cm}
\includegraphics[width=0.5\linewidth]{StabilityTypeII}
\caption{
$B_{CR}$, equilibrium consumer to resource biomass ratio $\hat{C}:\hat{R}$, and stability of the coexistence equilibrium as functions of temperature $T$ (plotted in Celsius) with (red and blue) and without (black) mass dependencies and the temperature-size rule.
Type-II functional response in blue.
Rate-constants were chosen to make $r = 2$, $K = 100$, $f(R) = 0.1$, $m = 0.6$, and $e = 0.15$ at 15$^\circ C$ (as in Figure 3 of \cite{Gilbert2014}).
Other parameters as in \cite{Gilbert2014}, \cite{DeLong2015}, and \cite{Rall2012}: $E_B = 0.32$, $E_S = 0.9$, $E_m = 0.65$, $E_{\nu,i} = 0.46$, $\nu_{0,i} = 1$, $\kappa = -0.81$, $\alpha = 1$, $\epsilon = -0.5$, $\mu = -0.29$, $\rho = -0.81$, $a_C = 1/4+2/3$, $a_R = 1/3$, $h_C = -2/3$, $h_R = 0.5$, $E_a = 0.65$, $E_h = -0.65$, $h_0 = 10^{-13}$.  
}
\label{TypeII}
\end{figure}

%%%%%%%%%%%%%%%%%%%%%%%%%%%%%%%%%%%%%%%%%%%%%%%%
%\begin{figure}[!ht]
%\centering
%\includegraphics[width=0.5\linewidth]{FunctionalResponseTemp}
%\caption{
%SUPPLEMENTARY FIGURE. Functional response as a function of temperature.
%Shown are a type-I functional response (black), a type-II functional response with temperature- and mass-dependencies of capture rate and handling time from \cite{Rall2012} (red), and a type-II functional response with temperature- and mass-dependencies of capture rate like that of attack rate in \cite{Gilbert2014} (blue).
%Rate-constants were chosen to make $r = 2$, $K = 100$, $a = 0.1$, $m = 0.6$, and $e = 0.15$ at 15$^\circ C$ \citep[as in Figure 3 of][] {Gilbert2014}.
%Other parameters as in \cite{Gilbert2014}, \cite{DeLong2015}, and \cite{Rall2012}: $E_B = 0.32$, $E_S = 0.9$, $E_m = 0.65$, $E_{\nu,i} = 0.46$, $\nu_{0,i} = 1$, $\kappa = -0.81$, $\alpha = 1$, $\epsilon = -0.5$, $\mu = -0.29$, $\rho = -0.81$, $\beta_i = 0.02$, $a_C = 1/4+2/3$, $a_R = 1/3$, $h_C = -2/3$, $h_R = 0.5$, $E_a = 0.65$, $E_h = -0.65$, $h_0 = 10^{-13}$.  
%}
%\label{FunctionalResponseTemp}
%\end{figure}

%%%%%%%%%%%%%%%%%%%%%%%%%%%%%%%%%%%%%%%%%%%%%%%%
%\begin{figure}[!ht]
%\centering
%\includegraphics[width=0.5\linewidth]{BCRAllTempMassDepAsymm}\\\vspace{-0.75cm}
%\includegraphics[width=0.5\linewidth]{CtoRAllTempMassDepAsymm}\\\vspace{-0.75cm}
%\includegraphics[width=0.5\linewidth]{StabilityAllTempMassDepAsymm}
%\caption{
%SUPPLEMENTARY FIGURE.
%$B_{CR}$, equilibrium consumer to resource biomass ratio $\hat{C}:\hat{R}$, and stability of the coexistence equilibrium as functions of temperature $T$ (plotted in Celsius) without the temperature-size rule (black) with a symmetric temperature-size rule (red) and with an asymmetric temperature-size rule (blue).
%Rate-constants (e.g., $r_0$) were chosen to make $r = 2$, $K = 100$, $a = 0.1$, $m = 0.6$, and $e = 0.15$ at 15$^\circ C$ (as in \cite{Gilbert2014}).
%Other parameters as in \cite{Gilbert2014,DeLong2015,Forster2012}: $E_B = 0.32$ (solid dark), $E_B = 0.9$ (dashed and solid light), $E_S = 0.9$ (solid), $E_S = 0.32$ (dashed), $E_m = 0.65$, $E_{\nu,i} = 0.46$, $\nu_{0,i} = 1$, $\kappa = -0.81$, $\alpha = 1$, $\epsilon = -0.5$, $\mu = -0.29$, $\rho = -0.81$, $\beta_i = 0$ (black) , $\beta_i = 0.02$ (red), $\beta_R = 0.02$ and $\beta_C = 0.04$ (blue).  
%}
%\label{AllTempMassDepAsymm}
%\end{figure}

%%%%%%%%%%%%%%%%%%%%%%%%%%%%%%%%%%%%%%%%%%%%%%%%
%\begin{figure}[!ht]
%\centering
%\includegraphics[width=0.5\linewidth]{CaptureTSRAsymm}\\\vspace{-0.75cm}
%\includegraphics[width=0.5\linewidth]{HandlingTSRAsymm}
%\caption{
%SUPPLEMENTARY FIGURE.
%The sensitivity of capture rate and handling time to temperature with (blue) and without (black) the temperature-size rule.
%Plotted are (A) $\log(b/b0)$ and (B) $\log(h/h0)$.
%Parameters as in \cite{Rall2012}: $a_C = 1/4+2/3$, $a_R = 1/3$, $h_C = -2/3$, $h_R = 0.5$, $E_a = 0.65$, $E_h = -0.65$.
%}
%\label{CaptureTimeT}
%\end{figure}

\end{document}
